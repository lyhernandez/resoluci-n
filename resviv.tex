\documentclass[11pt,letterpaper]{article}
\usepackage[utf8]{inputenc}
\usepackage[spanish]{babel}
\usepackage[left=1.5cm, right=1.5cm, top=1.5cm, bottom=1.5cm]{geometry}
\usepackage{graphicx}
\usepackage[T1]{fontenc}
\usepackage{helvetic}
\usepackage{titlesec}

\author{Dirección Municipal de la Vivienda de Gibara}
\title{Resolución}
\date{\today}

\titleformat{\section}[runin]
{\scshape\bfseries}
{}
{0em}
{}[:\quad]

\begin{document}
    \begin{figure}[t]
        \raggedright
        \def\svgwidth{0.85cm}
        \input{Coat_of_arms_of_Cuba.pdf_tex}
    \end{figure}
    \setlength{\textfloatsep}{2.5pt}
    \begin{flushleft}
        \textbf{\textsc{Directora de la dirección municipal de la vivienda Gibara}} \\ 
        %\vspace{0.05cm}
        \textbf{\textsc{Resolución No.}}
        \underline{\hspace {2cm}}  \\           
    \end{flushleft}
    \vspace{-2em}
    \section{POR CUANTO}
    La Ley No. 65/1988  Ley General de la Vivienda de fecha 23 de Diciembre de 1988, en el Artículo 122. 1 inciso a), tal como quedó modificada por el Decreto-Ley 322/2014, regula que las Direcciones Municipales de la Vivienda tendrán competencia para conocer y resolver las reclamaciones de derechos y el cumplimiento de las obligaciones sobre transferencia de propiedad que se deriven de las regulaciones contenidas por la citada Ley.
    \section{POR CUANTO}
        La propia Ley en el Artículo 126, modificado por el Decreto-Ley No. 322 de  31 de Julio del 2014 establece el procedimiento para la presentación de las reclamaciones de derecho ante las Direcciones Municipales de la Vivienda y la radicación de las mismas, así como los Artículos 127 y 128 del citado texto legal en cuanto al procedimiento para la tramitación y solución de estas por la mencionada instancia administrativa.
    \section{POR CUANTO}
        La citada Ley,  en el Artículo 81 Apartado 1 tal como quedó modificado por el Decreto-Ley No. 288 de 28 de octubre 2011, establece que la vivienda de residencia permanente cuyo titular haya salido definitivamente del país, es confiscada por el Estado al efecto de poder transmitir su propiedad a las personas con derecho a ello.    
    \section{POR CUANTO}
        La mencionada Ley General en el Artículo 81 Apartado 2 inciso a), modificado por el citado Decreto Ley establece que en el caso anterior tienen derecho a la transmisión gratuita de la propiedad de dichas viviendas: los cónyuge e hijos que concurren con el mismo derecho.
    \section{POR CUANTO}
        La Resolución No. V-001-2014 “Reglamento Complementario a la Ley General de la Vivienda” de fecha 27 de agosto del 2014 emitida por el Ministro de la Construcción, en su Capítulo III establece los requisitos y el procedimiento para la trasmisión gratuita de la propiedad de las viviendas en los casos de ausencia definitiva del país de sus propietarios.
    \section{POR CUANTO}
        La Ley No.142 del Proceso Administrativo de fecha 28 de Octubre del 2021 del Presidente de la República, vigente a partir del 1ro de Enero del 2022 regula el proceso para el conocimiento por los Tribunales de las pretensiones en relación con los actos administrativos en el ejercicio de sus funciones.
    \section{POR CUANTO}
        La Ley No. 113  “Del Sistema Tributario” promulgada en fecha 23 de Junio del 2012, en su Título V, CAPITULO UNICO, SECCION PRIMERA, Artículo 196, establece un impuesto que grava las transmisiones de bienes inmuebles, que se dispongan mediante Resolución Administrativa, disponiendo además, en el Artículo 197 que son actos jurídicos gravados por este impuesto las transmisiones de propiedad sobre bienes inmuebles.
    \section{POR CUANTO}
        La Dirección Municipal de la Vivienda de Gibara radicó el Expediente No. 35 de fecha 28 de Enero del 2022 a solicitud de Ana Iris Columbié Aldana, ciudadana cubana, natural de Sagua de Tánamo, mayor de edad, ama de casa, de estado conyugal divorciada, con CI. No. 63051519633, con domicilio legal en Calle 14 No. 13-A, entre 26 de Julio y 2 de Diciembre, Municipio Gibara, Provincia Holguín, interesó  la confiscación de la casa ubicada en la Calle 14 No. 13-B, entre las Calles 26 de Julio y 2 de Diciembre, Municipio Gibara, Provincia Holguín, la que es propiedad del señor  Yeinnier Galbán Columbié,  según la Resolución No. 420 de fecha 11 de Julio del 2005, dictada en el Expediente No. 359/05 por la Dirección Municipal de la Vivienda, de la que resulta ser madre, con motivo de la salida  definitiva  del  territorio  nacional  del  mismo  y  en  consecuencia  se  le reconozca el derecho a la transmisión gratuita de la propiedad de la vivienda, fundamentando su pretensión en el hecho que ocupa la misma y por el vínculo de parentesco antes mencionado.    
    \section{POR CUANTO}
        En la sustanciación del Expediente se comprobó que la vivienda sita en la Calle 14 No. 13-B, entre las Calles 26 de Julio y 2 de Diciembre, Municipio Gibara, Provincia Holguín, es propiedad de Yeinnier Galbán Columbié, según la Resolución No. 420 de fecha 11 de Julio del 2005, dictada en el Expediente No. 359/05 por la Dirección Municipal de la Vivienda,  constatándose que por Resolución No. 395/05 de 25 de Junio del 2005 dictada en el Expediente No. 395/05 se le concedió al referido titular el Derecho perpetuo de superficie por el que abonó la suma de \$ 607.20 MN, según se acredita en el comprobante de pago de fecha 7 de Junio del 2005 por el Banco Popular de Ahorro de Gibara y que mediante la Resolución No. 419 de 26 de Noviembre del 2010 dictada en el Expediente 343 por esta propia Dirección se le reconoció una diferencia de terreno de 157.72 metros cuadrados, por haber abonado la suma de \$ 1 261.76 MN que correspondía a la superficie total de 233.60 metros cuadrados, resultando probado que el mismo salió del Territorio Nacional el 26 de Diciembre del 2016, con destino a España, y que se considera emigrado a partir del 26 de Noviembre del 2019,  según se acredita en la Certificación expedida el 4 de Febrero del 2022 por 1er Teniente Caridad Torres Sosa, Jefe de Unidad  de Trámites del MININT Gibara.   
    \section{POR CUANTO}
        En las actuaciones que obran en el referido Expediente resulta probado que la mencionada vivienda se encuentra ocupada por la señora Ana Iris Columbié Aldana,  madre del emigrante Yeinnier Galbán Columbié, parentesco que se acredita mediante la Certificación de Nacimiento No. 11131445120000191967, expedida el 1 de Febrero  del 2022 por el Registro del Estado Civil de Moa, probándose el grado de parentesco expuesto por la misma para la fundamentación de su solicitud.    
    %\newpage
    \section{POR CUANTO}
        En el Dictamen de Descripción, Tasación, Medidas y Linderos No.908 de fecha 10 de Junio del 2019 expedido por la Dirección Municipal de Planificación Física de Gibara, se acredita la descripción de la casa de referencia, que demuestra las características técnicas constructivas valor legal y los requisitos de habitabilidad del inmueble de referencia.
    \section{POR CUANTO}
        Mediante la Resolución No. 23 de fecha 8 de marzo del 2021, emitida por la Directora de la Dirección Provincial de la Vivienda de Holguín fue promovida con carácter definitivo quien suscribe al cargo de Directora de la Dirección Municipal de la Vivienda de Gibara.
    \section{POR TANTO}
        En el ejercicio de las facultades que me están conferidas.
    \begin{center}
        \textbf{RESUELVO:}
    \end{center}

    \section{PRIMERO}
        Declarar CON LUGAR la reclamación presentada por la promovente, en mérito de los fundamentos de hechos y de derechos expuestos en la presente Resolución.
    \section{SEGUNDO}
        Confiscar a favor del Estado Cubano la vivienda ubicada en la Calle 14 No. 13-B, entre las Calles 26 de Julio y 2 de Diciembre, Municipio Gibara, Provincia Holguín,  propiedad de Yeinnier Galbán Columbié, y en consecuencia dejar sin efecto legal la la Resolución No. 420 de fecha 11 de Julio del 2005, dictada en el Expediente No. 359/05 por la Dirección Municipal de la Vivienda.
    \section{TERCERO}
        Transferir a ANA IRIS COLUMBIÉ ALDANA, sin que medie pago alguno, la propiedad de la casa ubicada en la Calle 14 No. 13-B, entre las Calles 26 de Julio y 2 de Diciembre, Municipio Gibara, Provincia Holguín, la que posee la descripción:\\
    \noindent
        Urbana: Casa de una planta, ubicada en terreno, construida en el año 2005, remodelada  y ampliada en el año 2010, de estado técnico bueno, con paredes de ladrillos y bloques, cubierta de losa de hormigón armado y pisos de cemento pulido y cemento,  compuesta de: área pavimentada al frente, sala-comedor, cocina, tres dormitorios, dos pasillos interiores, servicio sanitario, cuarto de baño, terraza, patio de servicio, pasillo exterior pavimentado, y patio de tierra,  con una superficie útil de 143.47 metros cuadrados, una superficie ocupada en terreno de 180.89 metros cuadrados,  de las cuales 162.12 metros cuadrados corresponden a la casa y 18.77 metros cuadrados a las áreas pavimentadas. Posee un valor legal de \$ 16 973.24 MN.\\
        
        El terreno que ocupa limita por el frente con la Calle 14 y casa No. 11 de esa propia calle, por donde de izquierda a derecha en una primera etapa mide 2.95 metros, en una segunda etapa en ángulo aproximado de 90º entrando mide 17.60 metros, en una tercera en ángulo aproximado de 90º entrando mide 6.05 metros, en una cuarta etapa en ángulo aproximado de 90º saliendo recto hasta el frente mide 17.60 metros, en una quinta y última etapa en ángulo aproximado de 90º saliendo mide 6.15 metros, por el lateral derecho saliendo limita con la casa No. 13 de la mencionada calle, por donde de fondo a frente en una primera etapa mide 16.30 metros, en una segunda etapa en ángulo aproximado de 90º entrando mide 11.20 metros, en una tercera y última etapa en ángulo aproximado de 90º saliendo, recto hasta el frente mide 19.10 metros, por el lateral izquierdo saliendo limita con la casa No. 1-F de la calle 16 y con la casa No. 9 de la Calle 14, por donde de fondo a frente en una primera etapa mide 9.30 metros, en una segunda etapa  en ángulo  aproximado de 90º saliendo mide 14.75 metros, en una tercera y última etapa en ángulo aproximado de 90º recto hasta el frente mide 24.85 metros y por el fondo limita con la casa No.4 F de la Calle 18, por donde mide 12.50 metros, con una superficie total de 496.27 metros cuadrados.
    \section{CUARTO}
        La presente Resolución constituye a todos los efectos legales Título de Propiedad de la referida vivienda, declarando que la titular disfrutará de todos los derechos y obligaciones que como propietaria le corresponde de acuerdo a las regulaciones establecidas en la Ley 65 Ley General de la Vivienda tal como quedó modificada por el Decreto Ley No. 288 del 28 de Octubre del 2011 y por el Decreto Ley No. 322/2014.
    \section{QUINTO}
        Disponer la obligación del pago del precio correspondiente a la diferencia de medidas del terreno descripto anteriormente ante la Dirección Municipal de Planificación Física de Gibara.
    \section{SEXTO}
        Advertirle a la persona antes mencionada que deberá solicitar la inscripción de éste Título de Propiedad en el Registro de la Propiedad de Gibara, debiendo previamente abonarse el Impuesto correspondiente para la transmisión de la propiedad de esta inmueble, a cuyos efectos deberán personarse en la Oficina Nacional de la Administración Tributaria (ONAT) de Gibara.\\ \par
    \noindent
    \textbf{NOTIFÍQUESE} a Lázaro Yasmani Hernández Galbán a nombre y representación de Ana Iris Columbié Aldana, haciéndole saber que en caso de inconformidad podrá establecer reclamación ante la Sección de lo Administrativo del Tribunal Municipal Popular de Holguín en el término de los cuarenta y cinco días hábiles siguientes a la notificación de la presente Resolución.\\ \par
    \noindent
    \textbf{ARCHÍVESE} la Resolución original en el Protocolo de Resoluciones de la Subdirección Legal de la Dirección Municipal de la Vivienda de Gibara.

    \begin{flushleft}
        \textrm{Gibara \underline{\hspace {1cm}} de \underline{\hspace {3cm}} de \underline{2022}.} \\ 
        \textit{“Año 64 de la Revolución”.}\\
                           
    \end{flushleft}
    \vspace{3cm}
    \noindent
    \textbf{Carmen Villafruela Fernández}

\end{document}